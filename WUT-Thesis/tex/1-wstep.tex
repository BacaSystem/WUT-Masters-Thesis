\newpage % Rozdziały zaczynamy od nowej strony.
\section*{Cel Pracy} \label{ss:Cel Pracy}

\noindent
Dynamiczny rozwój technologii oraz systemów sztucznej inteligencji umożliwił
realizację zaawansowanych i skomplikowanych obliczeniowo zadań bezpośrednio na urządzeniach mobilnych.
Postęp w architekturze modeli, metodach kompresji oraz wydajności układów (CPU/NPU) sprawił,
że aplikacje korzystające z AI stały się powszechne w środowisku mobilnym.
Równolegle wielcy dostawcy usług chmurowych udostępniają najnowocześniejsze modele AI
oferujące zaawansowane możliwości o niespotykanej wcześniej jakości i skali.
Rozwiązania te niosą jednak ze sobą zależność od zewnętrznej infrastruktury oraz wiążą się
z kwestiami prywatności, bezpieczeństwa i opłat subskrypcyjnych.

Na tym tle szczególne znaczenie zyskuje kwestia automatycznego generowania opisów obrazów 
(image captioning), łączącej zaawansowane przetwarzanie obrazu z generowaniem języka naturalnego.
Technologia ta znajduje zastosowanie w wielu aspektach, od wsparcia osób z dysfunkcją wzroku,
przez automatyczną kategoryzację treści, aż po integrację z asystentami głosowymi.


W obliczu tych możliwości twórcy aplikacji mobilnych stają przed dylematem wyboru
między wdrażaniem lokalnych modeli sztucznej inteligencji a integracją z usługami chmurowych dostawców.

Celem niniejszej pracy jest przeprowadzenie kompleksowej analizy porównawczej modeli działających 
w warunkach lokalnych oraz chmurowych w kontekście generowania opisów obrazów na urządzeniach 
z systemem Android. Badanie obejmuje porównanie trzech modeli lokalnych uruchamianych przez środowisko
ONNX Runtime (Florence-2, ViT-GPT2, BLIP) oraz trzech usług chmurowych udostępnianych przez interfejs
API (OpenAI GPT-4o mini, Azure Computer Vision, Google Gemini 2.5 Flash Lite).

Analiza efektywności poszczególnych modeli opiera się na dwóch komplementarnych aspektach:
metrykach wydajnościowych (czas generowania odpowiedzi, zużycie pamięci RAM, pobór energii, koszty wywołań API)
oraz metrykach jakościowych (poprawność generowanych opisów mierzona wskaźnikami CIDEr, SPICE, METEOR, BLEU).
Zebrane w ten sposób dane posłużą do sformułowania popartych empirycznie rekomendacji doboru rozwiązań
sztucznej inteligencji w zależności od przeznaczenia projektowanej aplikacji mobilnej.

Do realizacji powyższych celów zaprojektowano i zaimplementowano platformę badawczą CaptionLab, która
umożliwia prowadzenie badań, zbieranie metryk oraz zautomatyzowane testy na dużych zbiorach danych.
Architektura platformy badawczej została opracowana z myślą o elastyczności i rozszerzalności
dzięki jednolitemu podejściu do implementacji poszczególnych modeli w formie osobnych providerów.

Ostatecznie praca dostarcza nie tylko analityczną wiedzę odnośnie strategii wdrażania systemów 
sztucznej inteligencji w środowisku mobilnym, ale także funkcjonalną platformę badawczą, która
może służyć jako fundament dla dalszych badań lub jako funkcjonalne narzędzie przy podejmowaniu
decyzji architektonicznych w projektowaniu aplikacji mobilnych.



\newpage
\section{Wstęp}\label{s:Wstep}

\noindent
Systemy sztucznej inteligencji w ostatnich latach przeszły znaczną ewolucję od naukowych koncepcji
do powszechnie wykorzystywanych narzędzi, które zrewolucjonizowały wiele dziedzin życia i przemysłu.
AI znalazła zastosowanie i zmieniła funkcjonowanie całych sektorów gospodarki, począwszy od technologii i
rozrywki przez dziennikarstwo, kończąc na finansach czy medycynie. Według danych Presedence Research globalny rynek AI
osiągnął wartość ponad 750 miliardów USD w 2025 roku, a prognozowany dalszy wzrost, przewiduje wzrost
do prawie 3.7 bilionów USD do roku 2034 \cite{ai_market_precedence_2025}.

W spektrum zastosowań sztucznej inteligencji szczególnie wyróżnia się zadanie automatycznego generowania
opisów obrazów (image captioning) jako jedno z bardziej złożonych wyzwań łączących przetwarzanie obrazu,
kategoryzację wizualną oraz analizę kontekstu z generowaniem języka naturalnego. Systemy image captioning,
oparte na architekturach encoder-decoder \cite{show_and_tell_vinyals2015}, znajdują zastosowanie w wielu obszarach,
od wsparcia osób z dysfunkcją wzroku, przez automatyczną kategoryzację treści, aż po integrację
z asystentami głosowymi.

Rozwój technologii uczenia maszynowego umożliwił znaczący postęp w dziedzinie image captioning.
Systemy AI zyskały nowe możliwości dzięki innowacyjnym architekturom modeli oraz narzędziom
ułatwiającym deweloperom wdrażanie zaawansowanych rozwiązań.
Modele oparte na architekturze transformerów wizyjnych (Vision Transformer, ViT)
wypierały tradycyjne sieci konwolucyjne (Convolutional Neural Networks, CNN), oferując lepsze możliwości 
rozpoznawania i kwalifikacji obrazów \cite{vit_dosovitskiy2021}. Pojawienie się technologii jednolitych 
formatów modeli takich jak ONNX (Open Neural Network Exchange) umożliwiło łatwiejszą wymianę i wdrażanie 
zaawansowanych modeli AI w środowiskach o ograniczonych zasobach \cite{onnx_icsme_openja2022}, podczas gdy 
popularyzacja sieci 5G oraz rozwój chmurowego przetwarzania danych pozwoliło na korzystanie z potężnych
modeli AI poprzez interfejsy API (Application Programming Interface), eliminując potrzebę lokalnego przechowywania
i przetwarzania danych \cite{edge_computing_shi2016}.

Jednym z najbardziej dynamicznie rozwijających się obszarów zastosowań sztucznej inteligencji
są aplikacje i systemy mobilne. Współczesne urządzenia mobilne, niegdyś pełniące funkcję jedynie
narzędzi komunikacji i prostej rozrywki, dziś wyposażone są w zaawansowane układy obliczeniowe
(Neural Processing Unit, NPU) umożliwiające przetwarzanie złożonych modeli AI bezpośrednio na urządzeniach. Według raportu z 2025 roku opublikowanego przez Statista na świecie jest ponad 7 miliardów
zarejestrowanych w sieci smartfonów, co przekłada się na niemal 80\% globalnej populacji \cite{smartphone_users_2025}.
Ta masywna baza użytkowników połączona z ciągle rosnącą mocą obliczeniową układów mobilnych sprawia,
że platformy mobilne stają się kluczowym obszarem dla wdrażania rozwiązań opartych na sztucznej inteligencji.


Wybór aplikacji mobilnych jako obszaru badań nad efektywnością modeli AI jest konsekwencją szeregu specyficznych uwarunkowań środowiskowych.
Po pierwsze, urządzenia mobilne charakteryzują się ścisłymi ograniczeniami sprzętowymi, takimi jak limitowana ilość pamięci RAM, mocy obliczeniowej
czy ograniczonym źródłem zasilania. Po drugie, urządzenia funkcjonują w warunkach zmiennej jakości połączeń sieciowych, co podważa niezawodność rozwiązań 
zależnych wyłącznie od infrastruktury chmurowych dostawców. Po trzecie, kwestia prywatności i bezpieczeństwa danych użytkownika nabiera szczególnego 
znaczenia w kontekście osobistych urządzeń z dostępem do bardzo wrażliwych danych. Czynniki te powodują, że architektura systemów AI na platformach mobilnych
musi uwzględniać precyzyjny dobór rozwiązań, miejsca przetwarzania danych oraz strategii zarządzania zasobami.

W kontekście tych ograniczeń zarysowuje się istotny kompromis między podejściem lokalnym a chmurowym.
Modele uruchamiane lokalnie oferują działanie bez połączenia z siecią,
przewidywalność czasu odpowiedzi oraz silniejszą ochronę prywatności. Z drugiej strony usługi chmurowe oferują dostęp 
do najnowszych, dużych generatywnych modeli bez konieczności ich instalacji, aktualizacji czy obciążania pamięci urządzenia
kosztem zależności od sieci, kosztów subskrypcyjnych i zewnętrznego przetwarzania danych. Badania Yuyi Mao i innych wykazały,
że wybór miejsca wykonywania obliczeń (edge vs. cloud) ma istotny wpływ na opóźnienie, zużycie energii
oraz przepustowość systemu \cite{edge_vs_cloud_mao2017}. W przypadku środowiska o ograniczonych zasobach, 
chcąc jednocześnie oferować możliwie jak najlepszą jakość usługi, wybór odpowiedniego rozwiązania staje się kluczowym zadaniem projektowym dla twórców aplikacji mobilnych.


Brak kompleksowych badań porównawczych utrudnia jednak podejmowanie świadomych decyzji architektonicznych.
Niniejsza praca ma na celu wypełnienie tej luki poprzez przeprowadzenie szczegółowej analizy wydajnościowej
i jakościowej reprezentatywnych modeli lokalnych i chmurowych do generowania opisów obrazów w środowisku
mobilnym Android.


\subsection{Wymagania projektowe}\label{ss:Wymagania Projektowe}

\noindent
Głównym celem badawczym niniejszej pracy było przeprowadzenie kompleksowej analizy efektywności 
lokalnych oraz chmurowych rozwiązań sztucznej inteligencji w zadaniu generowania opisów obrazów (image captioning) 
na platformie mobilnej Android oraz sformułowanie rekomendacji doboru strategii wdrożeniowej 
w zależności od specyfiki i przeznaczenia projektowanych aplikacji mobilnych.

Kompleksowe porównanie efektywności modeli sztucznej inteligencji
w badanym kontekście wymagało
zdefiniowania problemu badawczego, doboru reprezentatywnych modeli oraz opracowania metodologii badawczej 
umożliwiającej obiektywną analizę poszczególnych modeli poprzez analizę zebranych w trakcie badań metryk.

Nim przystąpiono do realizacji głównego celu badawczego i wybrano poszczególne rozwiązania modeli sztucznej inteligencji
do analizy, należało zgłębić temat image captioning'u, poznać sam mechanizm działania, a także zrozumieć architekturę poszczególnych systemów 
dostępnych na rynku. W tym celu przeprowadzono szczegółową analizę literatury oraz dokumentacji technicznej
dotyczącej zarówno klasycznych, jak i współczesnych podejść do zadania automatycznego generowania opisów obrazów.

Na bazie tej wiedzy dokonano selekcji reprezentatywnych modeli AI różniących się architekturą, podejściem do przetwarzania danych, a także miejscem wykonywania obliczeń (lokalne vs. chmurowe). 
Wybrane modele miały pokrywać szerokie spektrum dostępnych rozwiązań,
umożliwiając w ten sposób analizę technologii o różnych podejściach i możliwościach.

Następnie, mając wyselekcjonowane rozwiązania, zaprojektowano i zaimplementowano dedykowaną platformę badawczą przystosowaną do wdrażania
i testowania różnorakich modeli AI w zadaniu image captioning'u. Platforma ta, poza wsparciem dla uruchamiania różnorodnych modeli, musiała także umożliwiać 
zbieranie precyzyjnych statystyk niezbędnych do obiektywnej analizy efektywności poszczególnych rozwiązań.

Opracowanie metodologii badawczej obejmowało zdefiniowanie zestawu metryk wydajnościowych i jakościowych, 
procedur pomiarowych oraz protokołu testowego, pozwalającego na zebranie odpowiednich danych, które w dalszej kolejności 
będą służyć do obiektywnej oceny efektywności badanych modeli AI.

Z opracowaną metodologią i sformułowanymi wymaganiami projektowymi, przystąpiono do realizacji badań, przeprowadzając serię eksperymentów
badawczych, zbierając dane dotyczące parametrów dla wybranych lokalnych i chmurowych modeli sztucznej inteligencji.

Ostatnim i zarazem kluczowym etapem była wizualizacja i interpretacja uzyskanych wyników badań, identyfikacja kompromisów między wydajnością a jakością
generowanych opisów oraz sformułowanie rekomendacji doboru strategii wdrożeniowej w zależności od specyfiki i wymagań projektowanej aplikacji mobilnej.

Wyżej sformułowane wymagania zebrano i przedstawiono w formie poszczególnych celów badawczych, których realizacja prowadzi do osiągnięcia głównego celu pracy.
Każdy z wyróżnionych niżej celów odpowiada kolejnym rozdziałom pracy, w których szczegółowo opisano podejście do realizacji i osiągnięcia poszczególnych celów.

\vspace{0.3cm}
\noindent
\textbf{Cel główny (CG):} 

Przeprowadzenie kompleksowej analizy efektywności lokalnych oraz chmurowych rozwiązań sztucznej 
inteligencji w zadaniu generowania opisów obrazów na platformie mobilnej Android 
oraz sformułowanie rekomendacji doboru strategii wdrożeniowej w zależności od specyfiki i przeznaczenia 
projektowanych aplikacji mobilnych.

\vspace{0.3cm}
\noindent
\textbf{Cele szczegółowe:}

\begin{enumerate}[leftmargin=*, label=\textbf{C\arabic*.}, itemsep=3pt]
    \item Przedstawienie teoretycznych podstaw automatycznego generowania opisów obrazów, analizy 
    architektur modeli AI stosowanych w zadaniu image captioning'u oraz identyfikacja kluczowych różnic 
    między strategiami wdrażania lokalnego i chmurowego w kontekście aplikacji mobilnych.
    
    \item Zaprojektowanie i implementacja dedykowanej platformy badawczej \textquote{CaptionLab} umożliwiającej 
    przeprowadzenie badań porównawczych modeli lokalnych i chmurowych w środowisku Android z zapewnieniem 
    precyzji pomiarów oraz powtarzalności testów.
    
    \item Opracowanie metodologii badawczej obejmującej definicję zestawu metryk wydajnościowych i jakościowych, 
    procedur pomiarowych oraz protokołów testowych pozwalających na obiektywną ocenę efektywności badanych 
    modeli AI.
    
    \item Przeprowadzenie serii eksperymentów badawczych oraz zebranie danych dotyczących parametrów wydajnościowych 
    (czas inferencji, zużycie pamięci, pobór energii, koszty operacyjne) i jakościowych (metryki CIDEr, SPICE, 
    METEOR, BLEU) dla wybranych modeli lokalnych i chmurowych.
    
    \item Interpretacja uzyskanych wyników badań, identyfikacja kompromisów między wydajnością a jakością 
    generowanych opisów oraz sformułowanie rekomendacji doboru strategii wdrożeniowej w zależności od 
    specyfiki i wymagań projektowanej aplikacji mobilnej.
\end{enumerate}

\noindent
Dla tak zdefiniowanych celów badawczych sformułowano odpowiadające im problemy badawcze,
na które niniejsza praca ma udzielić odpowiedzi. Problemy te wyznaczają główny kierunek prowadzonej
analizy oraz jasno definiują zakres prowadzonych badań.

\vspace{0.3cm}
\noindent
\textbf{Problem główny (PG):} 

Które podejście do wdrażania systemów AI generujących opisy obrazów (lokalne 
czy chmurowe) jest bardziej efektywne w środowisku mobilnym oraz jakie kryteria powinny kierować wyborem strategii wdrożeniowej?

\vspace{0.3cm}
\noindent
\textbf{Problemy szczegółowe:}

Analogicznie do dekompozycji celu głównego na cele szczegółowe problem główny został rozłożony na pięć 
problemów szczegółowych bezpośrednio odpowiadających celom szczegółowym C1-C5.

\begin{enumerate}[leftmargin=*, label=\textbf{P\arabic*.}, itemsep=3pt]
    \item Jak działa zadanie automatycznego generowania opisów obrazów,
    czym charakteryzują się poszczególne architektury modeli AI
    oraz jakie są kluczowe różnice pomiędzy strategiami wdrażania lokalnego i chmurowego?
    
    \item Jak efektywnie mierzyć i porównywać różnorodne modele AI (lokalne i chmurowe) 
    w zadaniu image captioning'u w środowisku mobilnym Android przy zapewnieniu precyzji pomiarów, 
    powtarzalności i wiarygodności eksperymentów?
    
    \item Jak opracować odpowiednią metodologię badań obejmującą wybór metryk, procedur pomiarowych 
    oraz protokołów testowych, która umożliwi obiektywną ocenę efektywności badanych modeli AI?
    
    \item Do jakich wyników prowadzą przeprowadzone eksperymenty badawcze w zakresie parametrów
    wydajnościowych i jakościowych dla wybranych lokalnych i chmurowych modeli sztucznej inteligencji?
    
    \item Jakie wnioski i rekomendacje wynikają z interpretacji zebranych danych oraz jakie 
    kompromisy i kryteria decyzyjne powinny kierować wyborem strategii wdrożeniowej 
    w zależności od specyfiki projektowanej aplikacji mobilnej?
\end{enumerate}

\noindent
Sformułowane wyżej cele oraz problemy badawcze stanowiły podstawę do wyprowadzania 
hipotez badawczych, które określają przewidywane odpowiedzi na postawione pytania badawcze. 
Hipotezy te zostały skonstruowane w oparciu o dotychczasowy stan wiedzy oraz obserwacje dotyczące kompromisów
między wydajnością a jakością w systemach uczenia maszynowego i odpowiadają kolejno postawionym problemom badawczym.

\vspace{0.3cm}
\noindent
\textbf{Hipoteza wiodąca (HG):} 

Nie istnieje uniwersalnie optymalne rozwiązanie dla wszystkich zastosowań. Modele lokalne 
charakteryzują się przewagą w zakresie ochrony prywatności, niezależności od sieci i innych dostawców, 
podczas gdy modele chmurowe oferują wyższą jakość generowanych opisów kosztem zależności od połączenia 
sieciowego i opłat operacyjnych, co implikuje konieczność doboru strategii wdrożeniowej w oparciu 
o specyficzne wymagania projektowanej aplikacji.

\vspace{0.3cm}
\noindent
\textbf{Hipotezy szczegółowe:}

\begin{enumerate}[leftmargin=*, label=\textbf{H\arabic*.}, itemsep=3pt]
    \item Architektury modeli lokalnych charakteryzują się specjalizacją i optymalizacją pod kątem ograniczonych zasobów,
    wykorzystując techniki kompresji, kwantyzacji oraz uproszczone enkodery wizyjne,
    podczas gdy modele chmurowe bazują na multimodalnych architekturach o znacznie większej pojemności,
    co przekłada się na fundamentalne różnice w strategiach wdrożeniowych.
    
    \item Efektywny pomiar i porównanie poszczególnych modeli AI wymaga dedykowanej platformy badawczej 
    implementującej ujednolicony interfejs pomiarowy, automatyzującą zbieranie metryk oraz eliminującą 
    zmienność środowiskową.
    
    \item Kompleksowa ocena efektywności modeli image captioning wymaga zdefiniowania 
    zestawu metryk wydajnościowych oraz jakościowych, 
    a także opracowania procesu testowego zapewniającego spójność i powtarzalność pomiarów pozwalających na bezpośrednie porównanie działania modeli.
    
    \item Modele lokalne charakteryzują się wyższym zużyciem pamięci RAM i energii 
    ze względu na lokalne przetwarzanie i przechowywanie wag modelu, jednak potencjalnie 
    oferują szybsze działanie, niezależne od warunków sieciowych, podczas gdy modele 
    chmurowe wykazują dłuższe czasy odpowiedzi uzależnione od opóźnień sieciowych, 
    ale zapewniają wyższe wartości metryk jakościowych.
    
    \item Interpretacja wyników prowadzi do wniosku, że modele lokalne stanowią optymalny wybór 
    dla aplikacji wymagających działania offline, niskich opóźnień, ochrony prywatności 
    oraz przewidywalnych kosztów, podczas gdy modele chmurowe są preferowane w scenariuszach 
    priorytetyzujących maksymalną jakość opisów i minimalne wymagania 
    pamięciowe aplikacji, przy akceptacji zależności sieciowej i kosztów proporcjonalnych do użytkowania.
\end{enumerate}

\noindent
Realizacja postawionych celów oraz weryfikacja sformułowanych hipotez określa strukturę logiczną pracy,
a także wyznacza kierunek prowadzonych badań. Osiągnięcie tych celów i odpowiedzi na postawione pytania badawcze
prowadzi do wyznaczenia praktycznych rekomendacji dotyczących doboru strategii wdrożeniowej systemów sztucznej inteligencji
w zadaniu image captioning'u w środowisku aplikacji mobilnych.


\subsection{Metodologia badawcza}\label{ss:Metodologia badawcza}

\noindent
W celu zrealizowania postawionych celów badawczych oraz weryfikacji sformułowanych hipotez, 
zastosowano metodologię badawczą opartą na eksperymencie porównawczym z wykorzystaniem metod ilościowych. 
Wybór tego podejścia metodologicznego wynikał z charakteru postawionego problemu badawczego, 
który wymaga obiektywnego, mierzalnego porównania dwóch odmiennych strategii wdrożeniowych 
w zadaniu generowania opisów obrazów na platformie mobilnej Android.

Wybór metody eksperymentu porównawczego był uzasadniony kilkoma kluczowymi względami. Umożliwił 
bezpośrednie zestawienie ze sobą dwóch konkurencyjnych podejść w identycznych warunkach środowiskowych, 
co wyeliminowało wpływ zewnętrznych zmiennych. Pozwolił na kontrolowanie warunków eksperymentu poprzez 
standaryzację danych wejściowych, procedur pomiarowych oraz środowiska testowego. Zapewnił również 
powtarzalność badań poprzez precyzyjne zdefiniowanie protokołu testowego, którego szczegółowy 
opis przedstawiono w rozdziale \ref{s:Metodologia badan}. \textquote{Metodologia badań}.
Porównanie obejmowało dwa komplementarne wymiary analizy. Wymiar wydajnościowy koncentrował 
się na aspektach operacyjnych działania modeli w środowisku mobilnym - czasach odpowiedzi, zużyciu 
zasobów systemowych oraz kosztach eksploatacyjnych. Wymiar jakościowy odnosił się do 
poprawności i trafności generowanych opisów obrazów mierzonej uznanymi metrykami oceny tekstu 
w języku naturalnym.

Realizacja eksperymentu porównawczego wymagała zastosowania metod ilościowych, które w przeciwieństwie 
do metod jakościowych, pozwalają na obiektywną, numeryczną ocenę badanych parametrów oraz umożliwiają 
statystyczne porównanie wyników między różnymi modelami i strategiami wdrożeniowymi. Metody ilościowe 
obejmowały pomiary bezpośrednie parametrów mierzalnych w trakcie działania modeli, takich jak 
czasy inferencji w poszczególnych fazach przetwarzania, zużycie pamięci RAM oraz 
rozmiary modeli. Zastosowano również pomiary pośrednie dotyczące kosztów operacyjnych modeli chmurowych, 
wyznaczonych na podstawie oficjalnych cenników dostawców API oraz oszacowania pobór energii urządzenia. W zakresie oceny jakości generowanych 
opisów wykorzystano uznane w literaturze metryki NLP (Natural Language Processing): CIDEr, SPICE, METEOR oraz BLEU, z których 
każda ocenia inne aspekty jakości opisu - od zgodności semantycznej, przez pokrycie pojęć, 
po płynność językową.

Takie podejście pozwoliło na obiektywną ocenę różnic między modelami lokalnymi 
a chmurowymi, statystyczne porównanie wyników oraz 
wizualizację wyników w formie wykresów i tabel ułatwiających interpretację i weryfikację postawionych hipotez
poprzez porównanie przewidywanych wartości z rzeczywistymi pomiarami, a także sformułowanie 
konkretnych, mierzalnych rekomendacji dotyczących wyboru strategii wdrożeniowej w zależności 
od przeznaczenia aplikacji.

\clearpage
\subsection{Struktura Pracy}\label{ss:Struktura Pracy}

\noindent
Wskazane cele, problemy i hipotezy badawcze w sekcji \ref{ss:Wymagania Projektowe}. \textquote{Wymagania projektowe}
zdefiniowały podział pracy na poszczególne jednostki redakcyjne, 
z których każda odpowiada realizacji kolejnych celów badawczych oraz weryfikacji postawionych hipotez.

Rozdział \ref{s:Modele}. \textit{Modele i technologie AI do generowania opisów obrazów} przedstawia szczegółowy opis teoretycznych podstaw automatycznego generowania opisów obrazów, ewolucji architektur
oraz mechanizmów działania systemów image captioning'u. Zawarto w nim także omówienie kluczowych różnic między strategiami wdrażania lokalnego i chmurowego,
a także charakterystykę wybranych modeli sztucznej inteligencji wykorzystanych w badaniach.

Szczegóły architektury i implementacji platformy badawczej zostały opisane
w rozdziale \ref{s:Aplikacja badawcza CaptionLab}. \textit{Aplikacja badawcza CaptionLab}. W ramach pracy napisany został dedykowany system badawczy 
\textquote{CaptionLab}, zdolny do przeprowadzania
testów porównawczych między różnymi modelami AI w środowisku Android, jednocześnie zbierając potrzebne 
dane do dalszej analizy. 

W rozdziale \ref{s:Metodologia badan}. \textit{Metodologia badania} przedstawiono proces projektowania i przeprowadzania 
eksperymentów badawczych, metodologię pomiarów, opis wykorzystanych w analizie metryk, a także sposób wyznaczania 
metryk jakościowych oceniających generowane opisy obrazów, na podstawie popularnych metod oceny jakości generowanego tekstu przez AI.

Wszelkie dane zebrane podczas przeprowadzania badań zostały przedstawione w rozdziale \ref{s:Wyniki badan}. \textit{Wyniki badań},
aby na ich podstawie dokonać analizy i wysnuć odpowiednią interpretację na łamach rozdziału \ref{s:Interpretacja wynikow badan}. \textit{Interpretacja wyników badań}.

Rozdział \ref{s:Podsumowanie}. \textit{Podsumowanie} zawiera końcowe wnioski z przeprowadzonych badań oraz wskazuje kierunki dalszych badań w obszarze wdrażania systemów
image captioning'u  opartych o AI na platformach mobilnych.


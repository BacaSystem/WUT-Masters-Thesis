% \newpage % Rozdziały zaczynamy od nowej strony.
\clearpage
\section{Metodologia badań}\label{s:Metodologia badan}

\noindent
TODO | W celu przeprowadzenia obiektywnych i powtarzalnych badań porównawczych modeli AI do generowania opisów obrazów, opracowano szczegółową metodologię badawczą obejmującą projekt eksperymentów, wybór datasetu, defin icję metryk oraz procedury wykonania testów. Niniejszy rozdział opisuje wszystkie aspekty metodologiczne niezbędne do zrozumienia i replikacji przeprowadzonych badań....

\subsection{Scenariusz badawczy}\label{ss:Scenariusz badawczy}
\noindent
TODO | Badania zostały zaprojektowane jako kontrolowany eksperyment porównawczy z następującymi parametrami...


\subsection{Metryki wydajnościowe}\label{ss:Metryki wydajnosci}

\noindent
Zdefiniowano precyzyjne metody pomiaru kluczowych parametrów wydajności.

\subsubsection{Metryki czasowe}\label{sss:Metryki czasowe}

\textbf{1. Preprocessing Time (pre\_ms)}

\textbf{2. Inference Time (infer\_ms)}

\textbf{3. Postprocessing Time (post\_ms)}

\textbf{4. End-to-End Time (e2e\_ms)}

\subsubsection{Metryki pamięciowe}\label{sss:Metryki pamieciowe}

\textbf{1. Peak RAM Usage (ram\_peak\_mb)}

\textbf{2. Model Size (model\_size\_mb)}


\subsubsection{Metryki energetyczne}\label{sss:Metryki energii}

\textbf{Energy Consumption (energy\_mwh)}


\subsubsection{Metryki kosztowe}\label{sss:Metryki kosztow}

\noindent
TODO | Dla modeli chmurowych obliczany jest koszt wywołania API...


\subsection{Metryki jakościowe} \label{ss:Metryki jakosciowe}
\noindent
TODO | W celu oceny jakości generowanych opisów obrazów zastosowano standardowe metryki NLP powszechnie używane w zadaniach generowania tekstu....


\subsubsection{CIDEr, SPICE} \label{sss:CIDEr SPICE}
\noindent



\subsubsection{BLEU, METEOR} \label{sss:BLEU METEOR}
\noindent


\subsection{Procedura pomiarowa}\label{ss:Proces badawczy}
\noindent
TODO | Opis procedury pomiarowej, w tym konfiguracja środowiska testowego, liczba powtórzeń, sposób zbierania danych i ich analiza....

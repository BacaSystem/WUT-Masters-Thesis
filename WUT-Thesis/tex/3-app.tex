\newpage % Rozdziały zaczynamy od nowej strony.
\section{Aplikacja badawcza CaptionLab}\label{s:Aplikacja badawcza CaptionLab}
\noindent
W celu przeprowadzenia kompleksowych badań porównawczych modeli AI została zaprojektowana i zaimplementowana dedykowana aplikacja badawcza CaptionLab na platformę Android. Aplikacja umożliwia testowanie zarówno modeli lokalnych działających w środowisku ONNX Runtime, jak i modeli chmurowych dostępnych poprzez API, przy jednoczesnym zbieraniu szczegółowych metryk wydajnościowych. 

W tym rozdziale opisano proces projektowania i implementacji systemu, jego architekturę oraz kluczowe komponenty odpowiedzialne za generowanie opisów obrazów i pomiar parametrów wydajności.

\subsection{Architektura systemu}\label{ss:Architektura systemu}
\noindent
Aplikacja CaptionLab została zaprojektowana zgodnie z zasadami czystej architektury (Clean Architecture) oraz wykorzystuje wzorce projektowe ułatwiające rozszerzalność i testowanie kodu.

\noindent
Architektura wysokiego poziomu składa się z trzech głównych warstw:

\begin{verbatim}
+======================================================+
|           WARSTWA PREZENTACJI                        |
|  +------------------+  +------------------------+    |
|  |  MainActivity    |  |  BatchTestActivity     |    |
|  |  - Single image  |  |  - Automated testing   |    |
|  |  - UI controls   |  |  - Progress tracking   |    |
|  +------------------+  +------------------------+    |
+======================================================+
|           WARSTWA LOGIKI BIZNESOWEJ                  |
|  +-----------------------------------------------+   |
|  |         ProviderManager                       |   |
|  |  - Register providers                         |   |
|  |  - Manage lifecycle                           |   |
|  +-----------------------------------------------+   |
|  +------------------+  +------------------------+    |
|  | BenchmarkRunner  |  |  MetricsCollector      |    |
|  | - Orchestrate    |  |  - Timing              |    |
|  | - Batch tests    |  |  - Memory              |    |
|  | - Error handling |  |  - Energy              |    |
|  +------------------+  +------------------------+    |
|  +------------------+  +------------------------+    |
|  |  DataExporter    |  |  DatasetLoader         |    |
|  +------------------+  +------------------------+    |
+======================================================+
|           WARSTWA PROVIDEROW AI                      |
|  +-----------------------------------------------+   |
|  |      CaptioningProvider (interface)           |   |
|  +-----------------------------------------------+   |
|  +--------------+ +--------------+ +-----------+    |
|  | Florence2    | |  ViTGPT2     | |   BLIP    |    |
|  | Provider     | |  Provider    | |  Provider |    |
|  | (ONNX)       | |  (ONNX)      | |  (ONNX)   |    |
|  +--------------+ +--------------+ +-----------+    |
|  +--------------+ +--------------+ +-----------+    |
|  |   OpenAI     | |    Azure     | |  Gemini   |    |
|  |  Provider    | |   Provider   | |  Provider |    |
|  |  (REST API)  | |  (REST API)  | | (REST API)|    |
|  +--------------+ +--------------+ +-----------+    |
+======================================================+
\end{verbatim}

\subsubsection{Warstwa prezentacji}\label{sss:Warstwa prezentacji}
\noindent

\subsubsection{Warstwa logiki biznesowej}\label{sss:Warstwa logiki biznesowej}
\noindent

\subsubsection{Warstwa providerów AI}\label{sss:Warstwa providerow AI}
\noindent


\subsection{Implementacja providerów AI}\label{ss:Implementacja providerow AI}
\noindent

\subsection{System zbierania metryk}\label{ss:System zbierania metryk}
\noindent

\subsection{Moduł testów automatycznych????}\label{ss:Modul testow automatycznych}
\noindent

\subsection{Eksport i analiza danych}\label{ss:Eksport i analiza danych}
\noindent

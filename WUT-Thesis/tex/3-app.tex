% \newpage % Rozdziały zaczynamy od nowej strony.
\clearpage
\section{Aplikacja badawcza CaptionLab}\label{s:Aplikacja badawcza CaptionLab}
\noindent
W celu przeprowadzenia kompleksowych badań porównawczych wybranych modeli AI, zaprojektowano 
i zaimplementowano aplikację badawczą \textquote{CaptionLab} na system Android. Platforma ta umożliwia testowanie
zarówno modeli lokalnych działających w środowisku ONNX Runtime, jak i modeli chmurowych dostępnych poprzez dostarczane API dostawców.
W trakcie generowania opisów platforma jednocześnie zbiera szczegółowe metryki wydajnościowe
i eksportuje je w formacie wygodnym do późniejszej analizy i interpretacji.

W tym rozdziale opisano architekturę systemu badawczego, jego kluczowe komponenty odpowiedzialne za generowanie opisów obrazów 
i pomiar parametrów wydajności, a także mechanizm dostarczania modeli AI, system zbierania metryk oraz moduł testów automatycznych.

\subsection{Architektura systemu}\label{ss:Architektura systemu}
\noindent
Aplikacja CaptionLab została zaprojektowana zgodnie z zasadami czystej architektury (Clean Architecture) oraz wykorzystuje wzorce projektowe ułatwiające rozszerzalność i testowanie kodu.
System został podzielony na trzy główne warstwy: warstwę prezentacji odpowiedzialną za interfejs użytkownika, warstwę logiki biznesowej zarządzającą przepływem danych i koordynacją testów, 
oraz warstwę providerów AI implementującą konkretne modele sztucznej inteligencji. Taki podział zgodnie z zasadami separacji odpowiedzialności (Separation of Concerns) 
zapewnia modularność, łatwość rozbudowy oraz możliwość niezależnego testowania poszczególnych komponentów.

Schemat architektury wysokiego poziomu przedstawiono na rysunku \ref{fig:captionlab_architecture}, który ilustruje wzajemne relacje między poszczególnymi warstwami oraz przepływ danych w systemie.

\begin{figure}[!h]
    \centering
    \begin{tikzpicture}[
        node distance=0.6cm,
        layer/.style={rectangle, draw=black, very thick, minimum width=13cm, align=center},
        component/.style={rectangle, draw, thick, rounded corners=2pt, minimum width=3cm, minimum height=1cm, align=center, font=\small},
        widecomp/.style={rectangle, draw, thick, rounded corners=2pt, minimum width=6.5cm, minimum height=1cm, align=center, font=\small},
        smallcomp/.style={rectangle, draw, thick, rounded corners=2pt, minimum width=2.4cm, minimum height=0.75cm, align=center, font=\footnotesize},
        interface/.style={rectangle, draw, thick, dashed, minimum width=12.5cm, minimum height=0.65cm, align=center, font=\small},
        flow/.style={->, thick, >=stealth},
        dataflow/.style={->, thick, >=stealth, dashed}
    ]
        % WARSTWA PREZENTACJI
        \node[layer, fill=blue!10, minimum height=1.8cm] (pres_layer) at (0,0) {};
        \node[above=0.05cm of pres_layer.north, font=\bfseries] {WARSTWA PREZENTACJI};
        
        \node[component, fill=blue!25] (main_act) at (-3,0) {MainActivity\\{\scriptsize Single Test}};
        \node[component, fill=blue!25] (batch_act) at (3,0) {BatchTestActivity\\{\scriptsize Batch Tests}};
        
        % WARSTWA LOGIKI BIZNESOWEJ
        \node[layer, fill=orange!10, minimum height=5.2cm, below=2.2cm of pres_layer] (logic_layer) {};
        \node[above=0.05cm of logic_layer.north, font=\bfseries] {WARSTWA LOGIKI BIZNESOWEJ};
        
        % Górny rząd: ProviderManager
        \node[widecomp, fill=orange!30, below=0.5cm of logic_layer.north] (provider_mgr) {ProviderManager\\{\scriptsize Rejestracja}};
        
        % Środkowy rząd: MetricsCollector i BenchmarkRunner
        \node[component, fill=orange!25, below=0.7cm of provider_mgr, xshift=-3.5cm] (metrics) {MetricsCollector\\{\scriptsize Pomiary}};
        \node[component, fill=orange!25, below=0.7cm of provider_mgr, xshift=3.5cm] (benchmark) {BenchmarkRunner\\{\scriptsize Orkiestracja}};
        
        % Dolny rząd: MemoryMonitor, PowerMonitor, DataExporter
        \node[smallcomp, fill=yellow!20, below=0.6cm of metrics, xshift=-1.3cm] (memory_mon) {MemoryMonitor};
        \node[smallcomp, fill=yellow!20, below=0.6cm of metrics, xshift=1.3cm] (power_mon) {PowerMonitor};
        \node[smallcomp, fill=yellow!20, below=0.6cm of benchmark] (exporter) {DataExporter};
        
        % WARSTWA PROVIDERÓW 
        \node[layer, fill=green!8, minimum height=6cm, below=2.2cm of logic_layer] (provider_layer) {};
        \node[above=0.05cm of provider_layer.north, font=\bfseries] {WARSTWA PROVIDERÓW AI};
        
        % INTERFACE
        \node[interface, fill=purple!8, below=0.4cm of provider_layer.north] (interface_layer) {CaptioningProvider Interface};
        
        % Providery w dwóch kolumnach po 3 wiersze
        % Lewa kolumna (lokalne)
        \node[smallcomp, fill=cyan!25, below=0.6cm of interface_layer, xshift=-3cm] (florence) {Florence-2\\{\tiny ONNX}};
        \node[smallcomp, fill=cyan!25, below=0.55cm of florence] (vitgpt2) {ViT-GPT2\\{\tiny ONNX}};
        \node[smallcomp, fill=cyan!25, below=0.55cm of vitgpt2] (blip) {BLIP\\{\tiny ONNX}};
        
        % Prawa kolumna (chmurowe)
        \node[smallcomp, fill=lime!25, below=0.6cm of interface_layer, xshift=3cm] (openai) {OpenAI\\{\tiny REST API}};
        \node[smallcomp, fill=lime!25, below=0.55cm of openai] (gemini) {Gemini\\{\tiny REST API}};
        \node[smallcomp, fill=lime!25, below=0.55cm of gemini] (azure) {Azure Vision\\{\tiny REST API}};
        
        % FLOWS
        \draw[flow] (main_act) -- (provider_mgr);
        \draw[flow] ([xshift=-0.5cm]main_act.south) -- ++(0,-0.5) -| (metrics.north);
        \draw[flow] ([xshift=0.5cm]batch_act.south) -- ++(0,-0.5) -| (benchmark.north);
        \draw[flow] (batch_act) -- (provider_mgr);
        \draw[flow] (benchmark) -- (metrics);
        \draw[dataflow] (benchmark) -- (exporter);
        \draw[flow] (metrics) -- (memory_mon);
        \draw[flow] (metrics) -- (power_mon);

        \draw[flow] (provider_mgr) -- (interface_layer);
        
        \coordinate (if_bottom) at (interface_layer.south);
        \coordinate (left_start) at ([xshift=-0.5cm]if_bottom);
        \draw[flow] (left_start) |- (florence.east);
        \draw[flow] (left_start) |- (vitgpt2.east);
        \draw[flow] (left_start) |- (blip.east);
        \coordinate (right_start) at ([xshift=0.5cm]if_bottom);
        \draw[flow] (right_start) |- (openai.west);
        \draw[flow] (right_start) |- (gemini.west);
        \draw[flow] (right_start) |- (azure.west);
        
        % ANNOTATIONS
        \node[above=0.05cm of florence, font=\tiny, text=cyan!70!black] {Modele Lokalne};
        \node[above=0.05cm of openai, font=\tiny, text=lime!70!black] {Modele Chmurowe};
        
    \end{tikzpicture}
    \caption{Architektura wysokiego poziomu aplikacji badawczej CaptionLab}
    \label{fig:captionlab_architecture}
\end{figure}

\subsubsection{Warstwa prezentacji}\label{sss:Warstwa prezentacji}
\noindent

\subsubsection{Warstwa logiki biznesowej}\label{sss:Warstwa logiki biznesowej}
\noindent

\subsubsection{Warstwa providerów AI}\label{sss:Warstwa providerow AI}
\noindent

\subsection{Implementacja providerów AI}\label{ss:Implementacja providerow AI}
\noindent

\subsection{System zbierania metryk}\label{ss:System zbierania metryk}
\noindent

\subsection{Moduł testów automatycznych????}\label{ss:Modul testow automatycznych}
\noindent

\subsection{Eksport i analiza danych}\label{ss:Eksport i analiza danych}
\noindent

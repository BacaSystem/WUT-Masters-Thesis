\newpage % Rozdziały zaczynamy od nowej strony.
\section{Interpratacja wyników badań}\label{s:Interpretacja wynikow badan}

\noindent
TODO | Na podstawie przeprowadzonych eksperymentów i zebranych danych, można wyciągnąć kilka kluczowych wniosków dotyczących wydajności i jakości generowanych opisów obrazów przez różne modele AI, zarówno lokalne, jak i chmurowe....

\subsection{Porównanie modeli AI}\label{ss:Porownanie modeli AI}
\noindent
TODO | Analiza porównawcza wykazała istotne różnice między modelami lokalnymi a chmurowymi pod względem opóźnień, zużycia zasobów oraz kosztów operacyjnych....


\begin{table}[!h] \centering
\caption{Przykładowa tabela}
\begin{tabular}{|l|c|c|c|c|} \hline
\label{tab:example_table}
\textbf{Model} & \textbf{metric 1} & \textbf{metric 2} & \textbf{+} & \textbf{-} \\ \hline\hline
ViT-GPT2 (local)       & 8.4  & 2.1 & 4  & 15 \\ \hline
BLIP (local)           & 10.2 & 2.8 & 5  & 18 \\ \hline
Florence-2 (local)     & 12.6 & 3.4 & 6  & 24 \\ \hline
\hline
Azure Vision (cloud)   & 14.8 & 3.9 & 7  & 27 \\ \hline
Gemini Pro (cloud)     & 22.4 & 5.6 & 12 & 38 \\ \hline
GPT-4o Vision (cloud)  & 26.7 & 6.8 & 14 & 45 \\ \hline
\end{tabular}
\end{table}

\subsection{Wnioski}\label{ss:Wnioski}
\noindent
TODO | Na podstawie przeprowadzonych badań można wyciągnąć następujące wnioski....
\newpage % Rozdziały zaczynamy od nowej strony.
\section{Podsumowanie i wnioski}\label{s:Podsumowanie}

\noindent
Niniejsza praca podejmuje zagadnienie dotyczące analizy porównawczej modeli sztucznej inteligencji działających w 
warunkach lokalnych oraz chmurowych w zadaniu automatycznego generowania opisów obrazów na platformie mobilnej Android 
oraz optymalnego wyboru strategii wdrożeniowej systemów AI w aplikacjach mobilnych.

Realizacja tego celu wymagała zbudowania dedykowanej platformy badawczej CaptionLab, która umozliwia przeprowadzenie kompleksowych
badań porównawczych sześciu reprezentatywnych modeli AI, obejmujących trzy modele lokalne uruchamiane przez środowisko ONNX Runtime (BLIP, ViT-GPT2, Florence-2)
oraz trzy usługi chmurowe dostępne przez interfejsy API (Azure Computer Vision, Google Gemini 2.5 Flash Lite, OpenAI GPT-4o mini) w komplementarnych aspektach wydajnościowych i jakościowych.
Badania zostały przeprowadzone na zestawie stu wyselekcjonowanych obrazów z kolekcji COCO, obejmując łącznie 3,018 przebiegów inferencji z 99-procentowym wskaźnikiem sukcesu.

Przeprowadzone badania przyniosły szereg istotnych odkryć, które znacząco odbiegają od początkowych założeń i przewidywań. 
Najważniejszym wnioskiem jest stwierdzenie, że specjalizacja modelu do konkretnego zadania image captioning'u ma większe znaczenie
dla osiąganych wyników niż sama strategia wdrożeniowa. To kluczowe odkrycie ujawnia się we wszystkich aspektach przeprowadzonej analizy
i stanowi główny motyw przewodni dla pozostałych wniosków.

Proces weryfikacji postawionych hipotez badawczych dostarczył mieszanych rezultatów, częściowo potwierdzając początkowe założenia, a częściowo je obalając.
Największym zaskoczeniem okazały się wyniki dotyczące jakości generowanych opisów oraz czasów inferencji, które nie zawsze korelowały z przewidywaniami opartymi na strategii wdrożeniowej,
tym samym podważając pierwotne założenia badawcze w tych zakresach.

Uzyskane wyniki badań niosą ze sobą istotne implikacje dla praktyki projektowania aplikacji mobilnych wykorzystujących systemy sztucznej inteligencji.
Najważniejszą rekomendacją jest piorytetyzacja specjalizacji modelu nad strategią wdrożeniową. Twórcy aplikacji powinni w pierwszej kolejności poszukiwać 
modeli dedykowanych do konkretnego zadania, niezależnie od tego, czy są dostępne lokalnie czy w chmurze.

Dla aplikacji priorytetyzujących responsywność systemów rekomendowane jest wykorzystanie Azure Computer Vision jako najszybszego rozwiązania lub lokalnego ViT-GPT2 w scenariuszach wymagających działania offline. 
W przypadku aplikacji wymagających najwyższej jakości opisów lokalny BLIP stanowi optymalny wybór pomimo wyższych wymagań zasobowych.
Z perspektywy ekonomicznej modele lokalne oferują przewagę w scenariuszach wysokiej częstotliwości użytkowania ze względu na zerowe koszty operacyjne.
W rozwiązaniach wymagających maksymalnej niezawodności rekomendowane są modele lokalne osiągające stuprocentowy wskaźnik sukcesu przez eliminację zależności sieciowych.
Z kolei w przypadku aplikacji mobilnych o znacznych ograniczeniach zasobowych modele chmurowe oferują optymalną strategię minimalizacji zużycia pamięci i energii urządzenia.

Przeprowadzone badania charakteryzują się jednak pewnymi ograniczeniami, które należy uwzględnić przy interpretacji wyników. 
Pierwszym ograniczeniem jest zawężenie analizy do platformy Android oraz konkretnego zestawu modeli AI. Wybór ten, choć uzasadniony reprezentatywnością badanych rozwiązań, 
nie wyczerpuje pełnego spektrum dostępnych technologii.

Drugim istotnym ograniczeniem jest oparcie oceny jakościowej wyłącznie na automatycznych metrykach porównujących generowane opisy z referencyjnymi opisami z zestawu COCO. 
Podejście to może niedoszacowywać rzeczywistej użyteczności opisów, szczególnie w przypadku modeli generujących semantycznie poprawne, lecz stylistycznie odmienne opisy od referencji.
Pełna ocena jakości wymagałaby uzupełnienia o ewaluację ludzką uwzględniającą inne aspekty w tym przydatność oraz satysfakcję użytkownika końcowego. 
Takie badania mogą w znaczący sposób zmienić postrzeganie jakości generowanych opisów i wpłynąć na ostateczne rekomendacje dotyczące wyboru modeli.

Uzyskane wyniki otwierają kilka perspektywicznych kierunków dla przyszłych badań w obszarze efektywności systemów sztucznej inteligencji na platformach mobilnych. 
Potencjalnymi kierunkami rozwoju jest rozszerzenie zakresu badań o inne platformy, integrację nowszych technologii oraz pogłębienie analizy jakościowej o ludzką ewaluację.
Możliwym obszarem eksploracji są również hybrydowe strategie wdrożeniowe łączące zalety podejścia lokalnego i chmurowego oraz badanie wpływu technik optymalizacyjnych na efektywność modeli lokalnych.

Przeprowadzone badania jednoznacznie dowodzą, że nie istnieje uniwersalnie optymalne rozwiązanie dla wszystkich scenariuszy 
zastosowań systemów sztucznej inteligencji w zadaniu generowania opisów obrazów na platformach mobilnych. 
Wybór między strategią lokalną a chmurową powinien być podejmowany w oparciu o szczegółową analizę wymagań konkretnej aplikacji, 
uwzględniającą priorytety wydajnościowe, jakościowe, ekonomiczne oraz operacyjne.

Kluczowym odkryciem badań jest stwierdzenie, że specjalizacja modelu do konkretnego zadania ma większe znaczenie niż strategia wdrożeniowa. 
Modele dedykowane do image captioning'u osiągają lepsze wyniki niż uniwersalne modele językowe adaptowane do przetwarzania obrazów,
niezależnie od miejsca wykonywania obliczeń.

Praktyczne implikacje tych odkryć sugerują, że twórcy aplikacji mobilnych powinni koncentrować się na identyfikacji najbardziej 
wyspecjalizowanych rozwiązań dostępnych dla ich konkretnych przypadków użycia, traktując strategię wdrożeniową jako wtórny czynnik decyzyjny. 
Takie podejście może prowadzić do znacząco lepszych rezultatów w zakresie wydajności, jakości oraz efektywności kosztowej implementowanych systemów AI.

Uzyskane wyniki stanowią solidną podstawę dla podejmowania świadomych decyzji architektonicznych w projektowaniu aplikacji mobilnych
wykorzystujących sztuczną inteligencję oraz otwierają perspektywiczne kierunki dla dalszych badań w tym dynamicznie rozwijającym się obszarze technologii.